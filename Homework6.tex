\documentclass[12pt]{article}
 \usepackage[margin=1in]{geometry} 
\usepackage{amsmath,amsthm,amssymb,amsfonts}
\usepackage{centernot}
\usepackage[table]{xcolor}
\usepackage{algpseudocode}
\usepackage{algorithm}
\usepackage{graphicx}
\floatname{algorithm}{}
 
\newcommand{\N}{\mathbb{N}}
\newcommand{\Z}{\mathbb{Z}}
\renewcommand*{\proofname}{Solution}


\newenvironment{problem}[2][Problem]
{\begin{trivlist}
\item[\hskip \labelsep {\bfseries #1}\hskip \labelsep {\bfseries #2.}]}{\end{trivlist}}


\makeatletter
\renewcommand{\fnum@algorithm}{\fname@algorithm}
\makeatother

\begin{document}

 
\title{CSCI 432\\Homework 6}
\date{}
\maketitle

Assigned 11/26/2018, due by end of class (4:00 pm) on 12/03/2018. Marked questions (*) are graded for correctness. The remaining will be graded for effort. Please see the course website for details about expected effort.

You must follow the collaboration policy detailed on the course website. Please type solutions in an appropriate editor (\LaTeX, Word) so that I can review equations and proofs efficiently. 

\begin{problem}{1*}
Given a graph $G=(V,E)$, we want to color each node with one of three colors. An edge is \textit{satisfied} if the colors for each of its nodes are different. The goal is to maximize the number of satisfied edges.

\begin{itemize}
\item Design an algorithm that randomly assigns each node a color.

\item Show that the probability a given edge is satisfied is $\frac{2}{3}$.

\item Use the following fact to show that the expected number of edges satisfied by our algorithm is at least $\frac{2}{3}$OPT. 

Fact: For random variables $X$ and $Y$, $E[X+Y] = E[X]+E[Y]$.
\end{itemize}
\end{problem}

\begin{problem}{2}
Suppose you have a random number generator that outputs $1$ with probability $0 < p \le 1$ and $0$ otherwise. Using this as a subroutine, make an algorithm that outputs $1$ with probability $\frac{1}{2}$ and $0$ otherwise.
\end{problem}

\begin{problem}{3*}
Evaluate the course by using the CS survey system: 

https://www.cs.montana.edu/survey/

I will not see what you write until after final grades are submitted. Even then, it is annonymous, so please be honest. I can't improve unless you tell me what is wrong. Put in a screen shot showing it is completed for credit.
\end{problem}

\begin{problem}{4}
Show up and participate in the workshop session on 11/30.
\end{problem}
 


\end{document}